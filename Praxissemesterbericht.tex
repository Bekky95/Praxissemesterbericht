\documentclass[a4paper, 12pt]{article}
\usepackage[german]{babel} % deutsch, deutsche Rechtschreibung
\usepackage[utf8]{inputenc} % Unicode-Zeichensatz als Text-Quelle
\usepackage[T1]{fontenc} % Umlaute und deutsches Trennen
\usepackage{mathptmx} % Times New Roman, gewohnter Font
%\usepackage{uarial} % Setzt den gesamten Text in Arial
\usepackage{courier} % einen schickeren Schreibmaschinenfont
\usepackage[scaled=.95]{helvet} % was serifenloses, wenn gebraucht
\usepackage{graphicx} % wir wollen Bilder einfügen
\usepackage{xfrac} % schöne Brüche im Fließtext mit sfrac

% hypersetup selbst ergänzt
\usepackage[dvipsnames]{xcolor}
%basic usage example: \color{ForestGreen}
% Apricot, Aquamarine, Bittersweet, Black, Blue, BlueGreen, BlueViolet, BrickRed,
% Brown, BurntOrange, CadetBlue, CarnationPink, Cerulean, CornflowerBlue, Cyan,
% Dandelion, DarkOrchid, Emerald, ForestGreen, Fuchsia, Goldenrod, Gray, Green,
% GreenYellow, JungleGreen, Lavender, LimeGreen, Magenta, Mahogany, Maroon,
% Melon, MidnightBlue, Mulberry, NavyBlue, OliveGreen, Orange, OrangeRed, Orchid,
% Peach, Periwinkle, PineGreen, Plum, ProcessBlue, Purple, RawSienna, Red, 
% RedOrange, RedViolet, Rhodamine, RoyalBlue, RoyalPurple, RubineRed, Salmon,
% SeaGreen, Sepia, SkyBlue, SpringGreen, Tan, TealBlue, Thistle, Turquoise,
% Violet, VioletRed, White, WildStrawberry, Yellow, YellowGreen, YellowOrange

\usepackage{hyperref}
\hypersetup{
	colorlinks=true,
    linkcolor=RoyalPurple,
    filecolor=ForestGreen,      
    urlcolor=Fuchsia,
    citecolor=BlueViolet
}
% basic colors: red, green, blue, cyan, magenta, yellow, black, gray,
% white, darkgray, lightgray, brown, lime, olive, orange, pink, purple, teal, violet
  
\usepackage{listings} % Schöne Quellcode-Listings [minted wäre besser]

% meine eigenen:
\usepackage{titlesec}
%\usepackage{minted} %ganz zum Schluss auf minted wechseln, 
\usepackage{enumitem} % Hilfreich, um Aufzählungen anzupassen
\usepackage{array}
\usepackage{subcaption}
\usepackage{pdfpages}


\lstset{basicstyle=\sffamily, columns=[l]flexible, mathescape=true, 
  showstringspaces=false, numbers=left, numberstyle=\tiny}
\lstset{language=python} % und nur schöne Programmiersprachen ;-)
% und eine eigene Umgebung für Listings

\usepackage{float} % eigene Fließobjekte, kommen an beliebigen Stellen vor
\newfloat{listing}{htbp}{scl}[section] % Nummeriere je Abschnitt
\floatname{listing}{Listing} % listing ist ein Fließobjekt

% Auch wenn es anrüchig ist, man kann den Platz etwas mehr ausnützen
\usepackage[paper=a4paper,width=15cm,left=30mm,height=22cm]{geometry}
\usepackage{setspace}
\linespread{1.1} % nicht ganz anderthalbzeilig, nur ein bisschen mehr Platz - war 1.15
\setlength{\parskip}{0.4em} % kleiner Paragraphen(Absatz)-abstand
\setlength{\parindent}{0em} % im Deutschen Einrückung nicht üblich

% Seitenmarkierungen 
\usepackage{fancyhdr} % Schickere Header und Footer
\pagestyle{fancy}

% Zeichensatz für Header/Footer
\newcommand{\phv}{\fontfamily{phv}\fontseries{m}\fontsize{9}{11}\selectfont}
\fancyhead[L]{\phv \leftmark} % Kurztitel links oben
\fancyhead[R]{\phv \thepage} % rechts oben die Seitenzahl
\fancyhead[C]{\phv Praxissemesterbericht} % oben Mitte Beschreibung article
\fancyfoot[L]{\phv Hochschule Mannheim} % Institution links unten
\fancyfoot[C]{\ } % keine Seitenzahl unten Mitte
\fancyfoot[R]{\phv Medizintechnik} % Studiengang rechts unten

\usepackage{url} % wir wollen eine URL anzeigen


\usepackage{titlesec}

\begin{document}

\begin{titlepage}
	\centering
	%\vspace*{2cm} %Abstand nach oben

	% Logos:	
    \begin{minipage}{0.1\textwidth}
        \includegraphics[height=2.5cm]
        {Hochschule_Mannheim_logo.png}
    \end{minipage}
    \hfill 
    \begin{minipage}{0.25\textwidth}
        \includegraphics[height=5cm]
        {loewenstein_logo.png}
    \end{minipage}	

    \vspace{2.5cm} % Abstand nach den Bildern
	
	%Titel:
	{\Huge\bfseries Praxissemesterbericht \par}
    \vspace{2cm} % Abstand nach dem Titel

    % titlepage content
    \begin{tabular}{ r  l }
        \textbf{Autor:} & Rebekka Hahn \\
        \textbf{Matrikelnummer:} & 1921861 \\
        \textbf{Semester:} & NACHSCHAUEN. Semester \\
        \textbf{Studiengang:} & Medizintechnik \\
        \textbf{Beginn Praxissemester:} & 02.09.2024 \\
        \textbf{Ende Praxissemester:} & 28.02.2025 \\
        \textbf{Firma:} & Löwenstein medical \\
        \textbf{Betreuer:} & Patrick von Poblotzki, Christoph Elsner \\
    \end{tabular}
    
    \vfill % Abstand nach unten

    % Ort und Datum
    {\large Ludwigshafen am Rhein, \today \par}

\end{titlepage}

\newpage
\tableofcontents            % Inhaltsverzeichnis generieren
\newpage                    % Neue Seite für den Textteil

{\bfseries Selbständigkeitserklärung}\\ \\
Ich versichere, dass ich diesen PS-Bericht selbständig und nur unter Verwendung der angegebenen
Quellen und Hilfsmittel angefertigt habe. Die Stellen, an denen Inhalte aus den Quellen verwendet
wurden, sind als solche eindeutig gekennzeichnet. Die Arbeit hat in gleicher oder ähnlicher Form bei
keinem anderen Prüfungsverfahren vorgelegen. \\
\vspace{1.0cm} \\
\line(1,0){430} \\
Datum, Ort und Unterschrift\\

\newpage
\section{Einleitung}
Hier beginnt die Einleitung deines Berichts. Du erklärst das Thema und gibst eine Übersicht.
\cite{digilink177512-pschyrembel}

\section{Theorie}
In diesem Abschnitt erläuterst du die theoretischen Grundlagen.

\subsection{Erster Unterpunkt}
Hier gehst du ins Detail zu einem spezifischen Teil der Theorie.

\section{Methodik}
Erkläre hier die Methodik, die du für dein Projekt oder deine Forschung verwendet hast.

\section{Ergebnisse}
Präsentiere und diskutiere hier die Ergebnisse deines Berichts.

\section{Fazit}
Im Fazit fasst du alles zusammen und gibst einen Ausblick.

%  Literaturverzeichnis
\newpage
\bibliographystyle{unsrt} %unsrt
\bibliography{Praxissemesterbericht_bibliographie}

\end{document}
