\documentclass[a4paper, 12pt]{article}
\usepackage[german]{babel} % deutsch, deutsche Rechtschreibung
\usepackage[utf8]{inputenc} % Unicode-Zeichensatz als Text-Quelle
\usepackage[T1]{fontenc} % Umlaute und deutsches Trennen
\usepackage{mathptmx} % Times New Roman, gewohnter Font
%\usepackage{uarial} % Setzt den gesamten Text in Arial
\usepackage{courier} % einen schickeren Schreibmaschinenfont
\usepackage[scaled=.95]{helvet} % was serifenloses, wenn gebraucht
\usepackage{graphicx} % wir wollen Bilder einfügen
\usepackage{xfrac} % schöne Brüche im Fließtext mit sfrac

% hypersetup selbst ergänzt
\usepackage[dvipsnames]{xcolor}
%basic usage example: \color{ForestGreen}
% Apricot, Aquamarine, Bittersweet, Black, Blue, BlueGreen, BlueViolet, BrickRed,
% Brown, BurntOrange, CadetBlue, CarnationPink, Cerulean, CornflowerBlue, Cyan,
% Dandelion, DarkOrchid, Emerald, ForestGreen, Fuchsia, Goldenrod, Gray, Green,
% GreenYellow, JungleGreen, Lavender, LimeGreen, Magenta, Mahogany, Maroon,
% Melon, MidnightBlue, Mulberry, NavyBlue, OliveGreen, Orange, OrangeRed, Orchid,
% Peach, Periwinkle, PineGreen, Plum, ProcessBlue, Purple, RawSienna, Red, 
% RedOrange, RedViolet, Rhodamine, RoyalBlue, RoyalPurple, RubineRed, Salmon,
% SeaGreen, Sepia, SkyBlue, SpringGreen, Tan, TealBlue, Thistle, Turquoise,
% Violet, VioletRed, White, WildStrawberry, Yellow, YellowGreen, YellowOrange

\usepackage{hyperref}
\hypersetup{
	colorlinks=true,
    linkcolor=RoyalPurple,
    filecolor=ForestGreen,      
    urlcolor=Fuchsia,
    citecolor=BlueViolet
}
% basic colors: red, green, blue, cyan, magenta, yellow, black, gray,
% white, darkgray, lightgray, brown, lime, olive, orange, pink, purple, teal, violet
  
\usepackage{listings} % Schöne Quellcode-Listings [minted wäre besser]

% meine eigenen:
\usepackage{titlesec}
%\usepackage{minted} %ganz zum Schluss auf minted wechseln, 
\usepackage{enumitem} % Hilfreich, um Aufzählungen anzupassen
\usepackage{array}
\usepackage{subcaption}
\usepackage{pdfpages}


\lstset{basicstyle=\sffamily, columns=[l]flexible, mathescape=true, 
  showstringspaces=false, numbers=left, numberstyle=\tiny}
\lstset{language=python} % und nur schöne Programmiersprachen ;-)
% und eine eigene Umgebung für Listings

\usepackage{float} % eigene Fließobjekte, kommen an beliebigen Stellen vor
\newfloat{listing}{htbp}{scl}[section] % Nummeriere je Abschnitt
\floatname{listing}{Listing} % listing ist ein Fließobjekt

% Auch wenn es anrüchig ist, man kann den Platz etwas mehr ausnützen
\usepackage[paper=a4paper,width=15cm,left=30mm,height=22cm]{geometry}
\usepackage{setspace}
\linespread{1.1} % nicht ganz anderthalbzeilig, nur ein bisschen mehr Platz - war 1.15
\setlength{\parskip}{0.4em} % kleiner Paragraphen(Absatz)-abstand
\setlength{\parindent}{0em} % im Deutschen Einrückung nicht üblich

% Seitenmarkierungen 
\usepackage{fancyhdr} % Schickere Header und Footer
\pagestyle{fancy}

% Zeichensatz für Header/Footer
\newcommand{\phv}{\fontfamily{phv}\fontseries{m}\fontsize{9}{11}\selectfont}
\fancyhead[L]{\phv \leftmark} % Kurztitel links oben
\fancyhead[R]{\phv \thepage} % rechts oben die Seitenzahl
\fancyhead[C]{\phv Praxissemesterbericht} % oben Mitte Beschreibung article
\fancyfoot[L]{\phv Hochschule Mannheim} % Institution links unten
\fancyfoot[C]{\ } % keine Seitenzahl unten Mitte
\fancyfoot[R]{\phv Medizintechnik} % Studiengang rechts unten

\usepackage{url} % wir wollen eine URL anzeigen


\usepackage{titlesec}
%\usepackage{minted} %ganz zum Schluss auf minted wechseln, 
\usepackage{enumitem} % Hilfreich, um Aufzählungen anzupassen
\usepackage{array}

\begin{document}

\begin{titlepage}
	\centering
	%\vspace*{2cm} %Abstand nach oben

	% Logos:	
    \begin{minipage}{0.1\textwidth}
        \includegraphics[height=2.5cm]
        {Hochschule_Mannheim_logo.png}
    \end{minipage}
    \hfill 
    \begin{minipage}{0.33\textwidth}
        \includegraphics[height=5cm]
        {loewenstein_logo.png}
    \end{minipage}	

    \vspace{2.5cm} % Abstand nach den Bildern
	
	%Titel:
	{\Huge\bfseries Praxissemesterbericht \par}
    \vspace{2cm} % Abstand nach dem Titel

    % titlepage content
    \begin{tabular}{ r  l }
        \textbf{Autor:} & Rebekka Hahn \\
        \textbf{Matrikelnummer:} & 1921861 \\
        \textbf{Semester:} & 11. Semester \\
        \textbf{Studiengang:} & Medizintechnik \\
        \textbf{Beginn Praxissemester:} & 02.09.2024 \\
        \textbf{Ende Praxissemester:} & 28.02.2025 \\
        \textbf{Firma:} & Löwenstein medical \\
        \textbf{Betreuer:} & Patrick von Poblotzki, Christoph Elsner \\
    \end{tabular}
    
    \vfill % Abstand nach unten

    % Ort und Datum
    {\large Ludwigshafen am Rhein, \today \par}

\end{titlepage}

\newpage
\tableofcontents 
\newpage

{\bfseries Selbständigkeitserklärung}\\ \\
Ich versichere, dass ich diesen PS-Bericht selbständig und nur unter Verwendung der angegebenen
Quellen und Hilfsmittel angefertigt habe. Die Stellen, an denen Inhalte aus den Quellen verwendet
wurden, sind als solche eindeutig gekennzeichnet. Die Arbeit hat in gleicher oder ähnlicher Form bei
keinem anderen Prüfungsverfahren vorgelegen. \\
\vspace{1.0cm} \\
\line(1,0){430} \\
Datum, Ort und Unterschrift\\

\newpage
\section{Einleitung}\label{Einleitung} 
Dieser Bericht fasst die Erfahrungen und Tätigkeiten zusammen, die ich während meines Praxissemesters bei Löwenstein Medical am Standort Karlsruhe sammeln konnte. Als Familienunternehmen im Bereich der Medizintechnik entwickelt und vertreibt Löwenstein Medical spezialisierte Beatmungsprodukte. Der Standort Karlsruhe hat bei der Entwicklung den Schwerpunkt Schlaftherapie, digitale Therapiebegleitung und Telehealth. Während meines Semesters war ich in der Firmware-Abteilung tätig und habe an einem Projekt zur Entwicklung eines Medizingerätes mitgearbeitet. 

Ziel dieses Berichts ist es, Einblicke in die Arbeitsweise und die speziellen Anforderungen der Firmware-Entwicklung in der Medizintechnik zu geben und die praktischen Erfahrungen zusammenzufassen, die ich in diesem professionellen Umfeld sammeln konnte. 

\newpage
\section{Löwenstein Medical}\label{loewenstein}
Einleitender Satz über die Section einfügen \cite{loewenstein}

\subsection{Über die Firma} %find evtl besseren Terminus
%Gründung in 1986 in Bad Ems als Heinen & Löwenstein Bereich Neonatologie.
Löwenstein Gruppe:(Löwenstein Medical, Löwenstein Medical Innovation, 
Löwenstein Medical Technology, WILAmed GmbH, Löwenstein Medical Austria, 
Löwenstein Medical Schweiz, Löwenstein Medical Belgien, Löwenstein Medical France,
Löwenstein Medical Netherlands, Löwenstein Medical UK, Löwenstein Medical Shanghai,
Löwenstein Medical RUS, Löwenstein Medical Americas, Löwenstein Medical Australia, 
IfM Ingenieursbüro für Medizintechnik GmbH, 
GMV Gesellschaft für medizintechnische Versorgung GmbH)


\subsection{Produkte}\label{products}
%einfach nur Auflistung == Zwischenstand, geht besser

\hspace*{-2.5cm} % Tabelle nach links schieben
\begin{tabular}{|l|m{5cm}|m{8.5cm}|}
\hline
\multicolumn{3}{|c|}{\textbf{Produkte der Firma Löwenstein Medical}} \\
\hline
\hline 
\textbf{Produktart} & \textbf{Unter-Produktart} & \textbf{Modell(e)} \\ 
\hline 
Intensivbeatmungsgeräte &  & elisa 500, elisa 300, elisa 800VIT, elisa 800, elisa 600, Hurrikan 200 \\ 
\hline 
Atemmasken & Homecare-Nasalmasken &  \\ 
\hline
 & Homecare-Full-Facemasken & \\
\hline
 & Klinikmasken & \\
\hline 
außerklinische Beatmung & Beatmungsgeräte & LUISA, prisma VENT30-C, prisma VENT40,prisma VENT50, prisma VENT50-C,	LM Flow, EO-150, Vivo 55, Vivo 65 \\ 
\hline 
 & Atemgasbefeuchter & AITcon Gen2, HC 550, LM 2000, MR 810, prisma VENT AQUA, VENTIclick \\ 
\hline 
Monitoring & SIDS-Monitore & VG 2100, VG 3100, VG310 \\ 
\hline 
 & Pulsoximeter & Masimo RAD-8/8v, Nellcor SpO2-System \\ 
\hline 
Sauerstofftherapie & Sauerstoffkonzentratoren & EverFlo \\ 
\hline 
 & tragbare Sauerstoffkonzentratoren & Evergo \\ 
\hline 
Schlafatemtherapie & CPAP- und APAP-Geräte & prisma20A, prisma20C, prisma SMART, prisma SMART plus und prisma SMART max, prisma SOFT, prisma SOFT plus und prisma SOFT max \\ 
\hline 
 & BiLevel-S und ST-Geräte & prisma25S, prisma25S-C, prisma25ST, prisma30ST, prisma30ST-HFT \\ 
\hline 
 & ASV- und Titrationsgeräte & prismaCR, prismaLAB \\ 
\hline 
 & Atemgasbefeuchter & LM 2000, prismaAQUA, SOMNOaqua \\ 
\hline 
Software &  & Schlafdatenbank, prisma CLOUD, 	prisma JOURNAL \\ 
\hline 
Schlafdiagnostik & Polysomnographiesysteme & Samoa \\ 
\hline 
 & Polygraphiegeräte & Samoa, Scala \\ 
\hline 
Sekretmanagement & Sekretmobilisation & The Vest, Cough Assist E70 \\ 
\hline 
 & Absauggeräte & Allegra M30 \\ 
\hline 
\end{tabular} 


%natürlich nur schon welche die draußen sind
% https://hul.de/wp-content/uploads/2014/10/HL-Image-Brosch%C3%BCre_DE_28-Seiten_RZ_-Ansicht_.pdf
% https://loewensteinmedical.at/produkte/

\newpage

\subsection{prisma Smart}\label{prismaSmart}
Eine genauere Analyse des Schlafherapiegerätes prisma SMART


\subsection{Qualitätsmanagement}\label{Qualitätsmanagement}
MDR, IVDR, AIMDD, etc

\subsection{Meetings}\label{Meetings}
SCRUM - Erklären
% wichtig - LM macht es nicht ganz nach Vorschrift, klassische Strukturen auch drin
\\ 
\textbf{SCRUM}\\
Scrum ist ein systematischer Ansatz um Projekte strukturiert durchzuführen. Es soll die Teams bei der Lösung komplexer Probleme unterstützen indem Rollen, Regeln und Ereignisse definiert werden. Die zugrundeliegenden Prinzipien sind Empirie und Lean Thinking. 
„Empirie, die Erfahrung selbst und die auf Erfahrung beruhende Erkenntnis.  
 \cite{dorsch_empirie}"
 
Lean Thinking https://www.lean.org/lexicon-terms/lean-thinking-and-practice/
bei ruhigerer Lage mal analysieren % TODO
 

\cite{scrum2020}
% https://scrumguides.org/docs/scrumguide/v2020/2020-Scrum-Guide-German.pdf


\newpage
\section{Aufgaben}\label{Aufgaben}
In diesem Abschnitt werden die Aufgaben während des Praxissemesters grob geschildert. Aufgrund der wirtschaftlichen Relevanz der zugrundeliegenden Daten kann nur eingeschränkt auf spezifische Inhalte eingegangen werden.

\subsection{Dokumentationsautomatisierung}\label{Dokumentationsautomatisierung}
python woop woop

\subsubsection{Polarion}\label{polarion}
Polarion Software ist ein Teil der Siemens Company und begann 2004

\subsubsection{Regular Expression}\label{regularExpression}
Die re-Bibliothek in Python ermöglicht die Anwendung regulärer Ausdrücke (regular expression - regex) zur flexiblen und effizienten Textverarbeitung. Regex sind Muster, die gezielt nach Zeichenfolgen in Textdaten suchen und so vielfältige Datenoperationen ermöglichen. Mit der re-library können Funktionen wie search, match, findall und sub genutzt werden, um beispielsweise Texte zu durchsuchen, Muster zu ersetzen und Daten zu validieren.

Die re-Syntax bietet eine Vielzahl von Operatoren: . steht für ein beliebiges Zeichen, * und + geben Wiederholungen an, und durch [] sowie () können Gruppen und Sets definiert werden.

\begin{lstlisting}[language=Python, caption=Beispiel für Python-Code]
import re

txt = "Das ist ein string mit 123 Zahlen"
pattern = r"\d+" # alle Ziffern

# in txt wird das pattern mit "00" ersetzt
new_txt = re.sub(pattern, "00", txt)
# new_txt: Das ist ein string mit 00 Zahlen
\end{lstlisting} % code has been testet - works

\subsection{Library Adapter}\label{LibraryAdapter}
Eine Library durch eine aktuellere austauschen in C++

\subsubsection{MsgPack}\label{msgpack}
library MsgPack

\newpage
\section{Ergebnisse}
Präsentiere und diskutiere hier die Ergebnisse deines Berichts.

\newpage
\section{Fazit}
Im Fazit fasst du alles zusammen und gibst einen Ausblick.

%  Literaturverzeichnis
\newpage
\bibliographystyle{unsrt} %unsrt
\bibliography{Praxissemesterbericht_bibliographie}

\end{document}
