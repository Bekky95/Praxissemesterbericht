\documentclass[a4paper, 12pt]{article}
\usepackage[german]{babel} % deutsch, deutsche Rechtschreibung
\usepackage[utf8]{inputenc} % Unicode-Zeichensatz als Text-Quelle
\usepackage[T1]{fontenc} % Umlaute und deutsches Trennen
\usepackage{mathptmx} % Times New Roman, gewohnter Font
%\usepackage{uarial} % Setzt den gesamten Text in Arial
\usepackage{courier} % einen schickeren Schreibmaschinenfont
\usepackage[scaled=.95]{helvet} % was serifenloses, wenn gebraucht
\usepackage{graphicx} % wir wollen Bilder einfügen
\usepackage{xfrac} % schöne Brüche im Fließtext mit sfrac

% hypersetup selbst ergänzt
\usepackage[dvipsnames]{xcolor}
%basic usage example: \color{ForestGreen}
% Apricot, Aquamarine, Bittersweet, Black, Blue, BlueGreen, BlueViolet, BrickRed,
% Brown, BurntOrange, CadetBlue, CarnationPink, Cerulean, CornflowerBlue, Cyan,
% Dandelion, DarkOrchid, Emerald, ForestGreen, Fuchsia, Goldenrod, Gray, Green,
% GreenYellow, JungleGreen, Lavender, LimeGreen, Magenta, Mahogany, Maroon,
% Melon, MidnightBlue, Mulberry, NavyBlue, OliveGreen, Orange, OrangeRed, Orchid,
% Peach, Periwinkle, PineGreen, Plum, ProcessBlue, Purple, RawSienna, Red, 
% RedOrange, RedViolet, Rhodamine, RoyalBlue, RoyalPurple, RubineRed, Salmon,
% SeaGreen, Sepia, SkyBlue, SpringGreen, Tan, TealBlue, Thistle, Turquoise,
% Violet, VioletRed, White, WildStrawberry, Yellow, YellowGreen, YellowOrange

\usepackage{hyperref}
\hypersetup{
	colorlinks=true,
    linkcolor=RoyalPurple,
    filecolor=ForestGreen,      
    urlcolor=Fuchsia,
    citecolor=BlueViolet
}
% basic colors: red, green, blue, cyan, magenta, yellow, black, gray,
% white, darkgray, lightgray, brown, lime, olive, orange, pink, purple, teal, violet
  
\usepackage{listings} % Schöne Quellcode-Listings [minted wäre besser]

% meine eigenen:
\usepackage{titlesec}
%\usepackage{minted} %ganz zum Schluss auf minted wechseln, 
\usepackage{enumitem} % Hilfreich, um Aufzählungen anzupassen
\usepackage{array}
\usepackage{subcaption}
\usepackage{pdfpages}


\lstset{basicstyle=\sffamily, columns=[l]flexible, mathescape=true, 
  showstringspaces=false, numbers=left, numberstyle=\tiny}
\lstset{language=python} % und nur schöne Programmiersprachen ;-)
% und eine eigene Umgebung für Listings

\usepackage{float} % eigene Fließobjekte, kommen an beliebigen Stellen vor
\newfloat{listing}{htbp}{scl}[section] % Nummeriere je Abschnitt
\floatname{listing}{Listing} % listing ist ein Fließobjekt

% Auch wenn es anrüchig ist, man kann den Platz etwas mehr ausnützen
\usepackage[paper=a4paper,width=15cm,left=30mm,height=22cm]{geometry}
\usepackage{setspace}
\linespread{1.1} % nicht ganz anderthalbzeilig, nur ein bisschen mehr Platz - war 1.15
\setlength{\parskip}{0.4em} % kleiner Paragraphen(Absatz)-abstand
\setlength{\parindent}{0em} % im Deutschen Einrückung nicht üblich

% Seitenmarkierungen 
\usepackage{fancyhdr} % Schickere Header und Footer
\pagestyle{fancy}

% Zeichensatz für Header/Footer
\newcommand{\phv}{\fontfamily{phv}\fontseries{m}\fontsize{9}{11}\selectfont}
\fancyhead[L]{\phv \leftmark} % Kurztitel links oben
\fancyhead[R]{\phv \thepage} % rechts oben die Seitenzahl
\fancyhead[C]{\phv Praxissemesterbericht} % oben Mitte Beschreibung article
\fancyfoot[L]{\phv Hochschule Mannheim} % Institution links unten
\fancyfoot[C]{\ } % keine Seitenzahl unten Mitte
\fancyfoot[R]{\phv Medizintechnik} % Studiengang rechts unten

\usepackage{url} % wir wollen eine URL anzeigen



\begin{document}

\begin{titlepage}
	\centering
	%\vspace*{2cm} %Abstand nach oben

	% Logos:	
    \begin{minipage}{0.1\textwidth}
        \includegraphics[height=2.5cm]
        {Hochschule_Mannheim_logo.png}
    \end{minipage}
    \hfill 
    \begin{minipage}{0.33\textwidth}
        \includegraphics[height=5cm]
        {loewenstein_logo.png}
    \end{minipage}	

    \vspace{2.5cm} % Abstand nach den Bildern
	
%Titel:
	{\Huge\bfseries Praxissemesterbericht \par}
    \vspace{2cm} % Abstand nach dem Titel

    % titlepage content
    \begin{tabular}{ r  l }
        \textbf{Autor} & Rebekka Hahn \\
        \textbf{Matrikelnummer} & 1921861 \\
        \textbf{Semester} & 11. Semester \\
        \textbf{Studiengang} & Medizintechnik \\
        \textbf{Beginn Praxissemester} & 02.09.2024 \\
        \textbf{Ende Praxissemester} & 28.02.2025 \\
        \textbf{Firma} & Löwenstein medical \\
        \textbf{Betreuer} & Patrick von Poblotzki, Christoph Elsner \\
    \end{tabular}
    
    \vfill % Abstand nach unten

    % Ort und Datum
    {\large Ludwigshafen am Rhein, \today \par}

\end{titlepage}

\newpage
{\bfseries \large Selbständigkeitserklärung}\\ \\
Ich versichere, dass ich diesen PS-Bericht selbständig und nur unter Verwendung der angegebenen
Quellen und Hilfsmittel angefertigt habe. Die Stellen, an denen Inhalte aus den Quellen verwendet
wurden, sind als solche eindeutig gekennzeichnet. Die Arbeit hat in gleicher oder ähnlicher Form bei
keinem anderen Prüfungsverfahren vorgelegen. \\
\vspace{1.0cm} \\
\line(1,0){430} \\
Datum, Ort und Unterschrift\\

\newpage
{\bfseries \large Abkürzungsverzeichnis}\\
\begin{table}[h!]
\centering
\begin{tabular}{c | c}
\hline
\textbf{Abkürzung} & \textbf{Ausgeschrieben} \\ 
\hline 
APAP & automatic positive airway pressure \\
CPAP & continuous positive airway pressure \\
OSA & obstruktive Schlafapnoe \\ 
SBAS & schlafbezogenen Atmungsstörungen \\ 
PAP & positive airway pressure \\
TE & Tonsillektomie \\
UPPP & Uvulopalatopharyngoplastik \\
\end{tabular} 
\end{table}

\newpage
{\bfseries \large Abstract}\\
sudo make abstract 

\newpage
\tableofcontents 

\newpage
\section{Einleitung}\label{Einleitung} 
Dieser Bericht fasst die Erfahrungen und Tätigkeiten zusammen, die ich während meines Praxissemesters bei Löwenstein Medical am Standort Karlsruhe sammeln konnte. Als Familienunternehmen im Bereich der Medizintechnik entwickelt und vertreibt Löwenstein Medical spezialisierte Beatmungsprodukte. Der Standort Karlsruhe hat bei der Entwicklung den Schwerpunkt Schlaftherapie, digitale Therapiebegleitung und Telehealth. Während meines Semesters war ich in der Firmware-Abteilung tätig und habe an einem Projekt zur Entwicklung eines Medizingerätes mitgearbeitet. 

Ziel dieses Berichts ist es, Einblicke in die Arbeitsweise und die speziellen Anforderungen der Firmware-Entwicklung in der Medizintechnik zu geben und die praktischen Erfahrungen zusammenzufassen, die ich in diesem professionellen Umfeld sammeln konnte. 

\newpage
\section{Löwenstein Medical}\label{loewenstein}
Dieses Kapitel gibt einen umfassenden Überblick über die Entwicklung des Unternehmens, präsentiert ein Beispielgerät aus dem Bereich der Heim-Schlafatemtherapie und beleuchtet den wissenschaftlichen Hintergrund der obstruktiven Schlafapnoe. Dabei wird sowohl auf technische Innovationen als auch auf therapeutische Ansätze und regulatorische Aspekte eingegangen.

\subsection{Geschichte und Entwicklung}
Löwenstein Medical wurde 1986 in Bad Ems gegründet. Nach dem Einstieg von Reinhard Löwenstein bei der Firma Heinen entstand das Unternehmen Heinen + Löwenstein, das sich zunächst auf die Neonatologie (Lehre der Pathologie und Physiologie Neugeborener) spezialisierte. Im Jahre 1992 folgte die Erweiterung um den Bereich Schlafmedizin. Zwei Jahre später, 1994, wurde die Heinen + Löwenstein Medizinelektronik gegründet, mit einem Fokus auf Schlafdiagnostiksystemen.

Ein bedeutender Entwicklungsschritt erfolgte im Jahr 1999, als Löwenstein Medical seine Position als führender Anbieter für respiratorische Heimversorgung in Deutschland etablierte und die exklusiven Vertriebsrechte für Produkte von Respironics sicherte. Die strategische Partnerschaft mit Hamilton Medical im Jahr 2002 trug zur Erweiterung des Kompetenzbereichs in der Beatmungstechnologien bei.

In den folgenden Jahren baute das Unternehmen kontinuierlich seine Produktpalette und internationale Präsenz aus. Zwischen 2005 und 2006 erfolgte die Markteinführung der Anästhesiegeräte Leon plus und Leon sowie der Beatmungsgeräte Leoni 2 und Leoni plus, speziell für Früh- und Neugeborene. Die Einführung der Flüssigsauerstoff-Versorgungslogistik im Jahr 2008 und die Integration von SALVIA medical waren weitere Punkte für die Weiterentwicklung des Unternehmens.

Die internationale Expansion begann 2009 mit der Gründung von Löwenstein Medical Austria. In den folgenden Jahren entstanden Tochtergesellschaften in zahlreichen Ländern, darunter Belgien, Frankreich, China, Australien und den USA. 2013 wurde Weinmann Homecare Teil der Gruppe, die ab 2017 unter dem Namen Löwenstein Medical firmiert.

2014 wurde die High-End-Intensivbeatmungsserie elisa 800 und elisa 600 auf den Markt gebracht, gefolgt von den Turbinenbeatmungsgeräten elisa 300 und elisa 500 im Jahr 2019. Der Launch des außerklinischen Beatmungsgeräts LUISA im Jahr 2020 und des Neonatologiegeräts LEONIE 4 im Jahr 2023 erweitern das Produktsortiment. \cite{loewenstein}


\subsection{Produkte}\label{products} %TODO: Umformulieren

Löwenstein Medical bietet eine breite Palette an Produkten, die auf die Bedürfnisse der Patienten  in den Bereichen Beatmung, Schlafatemtherapie und Sauerstoffversorgung ausgerichtet sind.

Intensivbeatmungsgeräte:
Im Bereich der Intensivbeatmung gibt es die Geräte der elisa-Reihe. Besonders das elisa 800 wird aufgrund seiner fortschrittlichen Visualisierungsfunktionen und der hohen Präzision bei der Beatmung von Intensivpatienten geschätzt. Diese Geräte bieten nicht nur eine zuverlässige Versorgung, sondern auch eine benutzerfreundliche Handhabung für medizinisches Personal. Weitere Modelle wie das elisa 600 und elisa 300 bieten bewährte Technologien für den klinischen Einsatz.

Außerklinische Beatmung:
Für die außerklinische Beatmung bietet Löwenstein Medical das LUISA-Beatmungsgerät, das für die Heimbeatmung von Patienten konzipiert wurde. LUISA zeichnet sich durch seine kompakte Bauweise und einfache Handhabung aus, wodurch es besonders für den häuslichen Gebrauch geeignet ist. Auch die Geräte der prisma VENT-Serie, wie das prisma VENT30-C und prisma VENT50-C, bieten flexible Einsatzmöglichkeiten für die häusliche Beatmung und werden durch eine Vielzahl an Zubehör wie den AITcon Gen2 Atemgasbefeuchter ergänzt.

Schlaftherapie:
Ein weiteres Highlight im Portfolio von Löwenstein Medical sind die prisma SMART-Geräte für die Schlafapnoe-Therapie. Das prisma SMART ermöglicht eine flexible Anpassung der Druckeinstellungen im APAP-Modus, was eine komfortable und effektive Therapie bei obstruktiver Schlafapnoe gewährleistet. In dem folgenden Unterkapitel wird näher auf den prisma SMART und was mit diesem behandelt wird eingegangen.

Schlafdiagnostik und Monitoring:
Für die Diagnose von Schlafapnoe und anderen Schlafstörungen bietet Löwenstein Medical Polysomnographiesysteme wie das Samoa, das eine präzise Analyse des Schlafverhaltens ermöglicht. Für den Heimgebrauch oder eine vereinfachte Diagnose stehen auch Polygraphiegeräte wie Scala zur Verfügung, die eine effektive Überwachung der Schlafparameter ermöglichen.

Sekretmanagement:
Im Bereich des Sekretmanagements bietet Löwenstein Medical Geräte wie den Cough Assist E70, der Patienten dabei unterstützt, überschüssiges Sekret effektiv zu mobilisieren und zu entfernen. Dies ist besonders für Patienten mit neurologischen Erkrankungen von Bedeutung.

Zusätzlich umfasst das Produktportfolio von Löwenstein Medical verschiedene Atemmasken, Sauerstoffkonzentratoren, Pulsoximeter und Softwarelösungen wie prisma CLOUD, die eine einfache Verwaltung von Patienteninformationen ermöglichen.

\newpage
\subsection{Beispielgerät der Schlafatemtherapie}\label{prismaSmart}
Eine genauere Analyse des Schlafherapiegerätes prisma SMART

\subsubsection{Physiologie Schlaf}
%TODO: Recherche

\subsubsection{Pathophysiologie Schlafapnoe}\label{schlafapnoe}
Die obstruktive Schlafapnoe (OSA) gehört zu den schlafbezogenen Atmungsstörungen (SBAS) und ist eine Erkrankung die durch wiederholte Atempausen während des Schlafens gekennzeichnet ist. Es gibt vier verschiedene Phänotypen die eine OSA verursachen können:

5.1 Obstruktive Schlafapnoe
Entsprechend der ICSD-3 [10] wird eine
obstruktive Schlafapnoe (OSA) dann diagnostiziert, wenn die Atmungsstörung
durch keine andere Schlafstörung oder
medizinische Erkrankung oder durch
Medikamente oder andere Substanzen
erklärbar ist und entweder ein AHI >
15/h (Ereignis jeweils größergleich 10 s) Schlafzeit oder ein AHI größergleich 5/h Schlafzeit in
Kombination mit einer typischen klinischen Symptomatik oder relevanten
Komorbidität vorliegt. 
Tagesschläfrigkeit bis hin zum unfreiwilligen Einschlafen ist das führende klinische Symptom der obstruktiven Schlafapnoe --Hauptbefund
Nächtliches Aufschrecken mit kurzzeitiger Atemnot, Schnarchen (bei 95prozent der Betroffenen),  Isoliert betrachtet, weisen die Symptome jedoch nur eine geringe Spezifität auf  --Nebenbefund
--> Definition OSA

\begin{itemize}
\item \textbf{Anatomische Einschränkungen der oberen Atemwege:}\\
Hierbei handelt es sich um strukturelle Faktoren, wie eine Verengung der Atemwege durch vergrößerte Mandeln, Fettansammlungen oder andere anatomische Besonderheiten. Diese Phänotypen sind häufig bei übergewichtigen oder älteren Menschen zu beobachten.

\item \textbf{niedrige respiratorische Erregungsschwelle (Arousals):}\\
Eine niedrige Erregungsschwelle bedeutet, dass Personen während des Schlafs leichter durch Atemprobleme geweckt werden. Dies führt zu fragmentiertem Schlaf und verhindert eine kontinuierliche Atmung. Umgekehrt kann eine zu hohe Schwelle die Sauerstoffsättigung gefährlich abfallen lassen.

\item \textbf{Instabilität des Atemantriebs („Loop Gain“):}\\
Diese Phänotypen beschreiben Menschen, deren Atmungssystem zu Überreaktionen neigt, was zu wechselnden Phasen von Hyperventilation und Hypoventilation führt. Dies verstärkt das Auftreten von Atempausen und Sauerstoffmangel.

\item \textbf{schlechte Funktion der oberen Atemwegsmuskulatur:}\\
Hier liegt das Problem in einer unzureichenden Aktivierung oder Kontrolle der Muskeln, die die Atemwege während des Schlafs offenhalten sollten. Besonders während des REM-Schlafs, in dem der Muskeltonus generell abnimmt, kann dies zu Atemwegsblockaden führen.
\end{itemize}
Diese Phänotypen sind nicht immer isoliert, sondern treten oft in Kombination auf. Ein besseres Verständnis der individuellen Merkmale ermöglicht eine gezieltere Diagnostik und Therapie der OSA. So können gezielte Therapielösungen für die spezifischen Pathomechanismen entwickelt werden. 
\cite{OSA_Pathophysiology2019} \cite{DGSM2017}

\subsubsection{Therapie Schlafapnoe}\label{therapy}

Die Behandlung der obstruktiven Schlafapnoe (OSA) richtet sich nach dem individuellen Beschwerdebild, den Begleiterkrankungen sowie den persönlichen Bedürfnissen und dem Therapiewillen des Patienten. Ziel ist es, die schlafbezogenen Atmungsstörungen zu beseitigen, die Schlafqualität zu verbessern und das Risiko für kardiovaskuläre und andere Komplikationen zu senken. Abhängig von der Schwere der Erkrankung stehen verschiedene Therapieansätze zur Verfügung, die von konservativen Maßnahmen über apparative Unterstützung bis hin zu chirurgischen Eingriffen reichen. \cite{DAE81892} \cite{flexikon}

\begin{itemize}
    \item \textbf{apparative Therapie}\\
    Die Standardtherapie der obstruktiven Schlafapnoe ist die nächtliche Überdruckbeatmung („positive airway pressure“, PAP), als die konkrete Referenzmethode im kontinuierlichen PAP-Modus (CPAP, „continuous positive airway pressure“). Die Indikationsstellung zur CPAP-Therapie erfolgt anhand einer Kombination aus klinischer Anamnese, polysomnographischem Befund und Begleiterkrankungen. Besonders wenn ohne Therapie eine Verschlechterung dieser Erkrankungen zu erwarten ist, wird eine CPAP-Therapie empfohlen. Der Therapiewille des Patienten sowie dessen individuelle Situation spielen ebenfalls eine entscheidende Rolle.
    
    Neben CPAP kommt auch der APAP-Modus (automatisch titrierendes PAP) zum Einsatz, der den Atemwegsdruck flexibel an die Bedürfnisse des Patienten anpasst. Beide Ansätze zielen darauf ab, den Kollaps der oberen Atemwege zu verhindern und die Atmung während des Schlafes zu stabilisieren. Kontraindikation der APAP sind zentrale Atmungsstörungen, kardio- pulmonale Erkrankungen und nächtliche Hypoventilationen. Die APAP kommt vorallem zum Einsatz bei Patienten, die den kontinuierlichen Druck der CPAP nicht mehr ertragen, bei komplexen Apnoen oder mangelnder Compliance. 

    \item \textbf{konservative Therapie}\\
    Konservative Maßnahmen umfassen Lebensstiländerungen wie Gewichtsreduktion, die Vermeidung von Alkohol und Sedativa sowie das Einhalten einer guten Schlafhygiene. Bei lageabhängiger OSA kann Lagetraining, das eine Rückenlage vermeidet, ebenfalls hilfreich sein.

    \item \textbf{medikamentöse Therapie}\\
    Medikamentöse Ansätze spielen bei der Behandlung der OSA nur eine untergeordnete Rolle, da bisher keine Substanzen eine ausreichende Wirksamkeit gezeigt haben.

    \item \textbf{chirurgische Therapie}\\
    Operative Eingriffe werden nur bei spezifischen anatomischen Ursachen wie Tonsillenhyperplasie oder kraniofazialen Fehlbildungen in Betracht gezogen. Zu den Methoden zählen beispielsweise die Uvulopalatopharyngoplastik oder Kieferrekonstruktionen. 
    
    Die Uvulopalatopharyngoplastik mit Tonsillektomie nach Fujita, die TE-UPPP, ist ein chirurgisches Verfahren bei welchem die Atemwege durch Entfernung von überschüssigem Gewebe erweitert werden. Das Ziel des Eingriffs ist eine Verringerung der Kollapsneigung der oberen Atemwege während des Schlafens. Im Rahmen eines Case Reports war eine Indikation der OSA einen verengten oropharyngealen Raum mit vergleichsweise großer Uvula (Gaumenzäpfchen) und überschüssiger Mucosa (Schleimhaut) des umgebenden Gewebes. Die Tonsillektomie ist die vollständige chirurgische Entfernung der Tonsilla Palatina (Gaumenmandel). Anschließend wird redundante Mucosa entfernt und das Uvula korrigiert. \cite{Fujita_UPPP}
\end{itemize}


\subsubsection{prisma Smart}
NOCH SAMMELSURIUM AN TEXTFETZEN:
technisches, patienten ui, etc
SMART MACHT APAP Therapie

Tiefschlafindikator (prisma RECOVER), Zwei Dynamik-Optionen im APAP-Modus
druckkontrollierte, nicht-invasive, nichtlebenserhaltende Therapiegeräte zur Behandlung schlafbezogener Atmungsstörungen (SBAS) mittels Maske.
 Funktionsbeschreibung
Eine Turbine saugt Umgebungsluft über einen Filter an und befördert sie mit dem
Therapiedruck über das Schlauchsystem und dem Beatmungszugang zum Patienten. 
Im autoCPAP-Modus (prisma SMART) wird der Druck kontinuierlich innerhalb
einstellbarer Grenzen angepasst und der jeweils erforderliche Druck abgegeben, der
die oberen Atemwege offenhält. 
\cite{manual_smart}

\subsection{Qualitätsmanagement}\label{Qualitätsmanagement}
MDR, IVDR, AIMDD, etc

\subsection{Meetings}\label{Meetings}
SCRUM - Erklären
% wichtig - LM macht es nicht ganz nach Vorschrift, klassische Strukturen auch drin
\\ 
\textbf{SCRUM}\\
Scrum ist ein systematischer Ansatz um Projekte strukturiert durchzuführen. Es soll die Teams bei der Lösung komplexer Probleme unterstützen indem Rollen, Regeln und Ereignisse definiert werden. Die zugrundeliegenden Prinzipien sind Empirie und Lean Thinking. 
„Empirie, die Erfahrung selbst und die auf Erfahrung beruhende Erkenntnis.  
 \cite{dorsch_empirie}" 

something something 
\cite{scrum2020}

\newpage
\section{Aufgaben}\label{Aufgaben}
In diesem Abschnitt werden die Aufgaben während des Praxissemesters grob geschildert. Aufgrund der wirtschaftlichen Relevanz der zugrundeliegenden Daten kann nur eingeschränkt auf spezifische Inhalte eingegangen werden.

\subsection{Dokumentationsautomatisierung}\label{Dokumentationsautomatisierung}
python woop woop

\subsubsection{Polarion}\label{polarion}
Polarion Software ist ein Teil der Siemens Company und begann 2004

\subsubsection{Regular Expression}\label{regularExpression}
Die re-Bibliothek in Python ermöglicht die Anwendung regulärer Ausdrücke (regular expression - regex) zur flexiblen und effizienten Textverarbeitung. Regex sind Muster, die gezielt nach Zeichenfolgen in Textdaten suchen und so vielfältige Datenoperationen ermöglichen. Mit der re-library können Funktionen wie search, match, findall und sub genutzt werden, um beispielsweise Texte zu durchsuchen, Muster zu ersetzen und Daten zu validieren.

Die re-Syntax bietet eine Vielzahl von Operatoren: . steht für ein beliebiges Zeichen, * und + geben Wiederholungen an, und durch [] sowie () können Gruppen und Sets definiert werden.

\begin{lstlisting}[language=Python, caption=Beispiel für Python-Code]
import re

txt = "Das ist ein string mit 123 Zahlen"
pattern = r"\d+" # alle Ziffern

# in txt wird das pattern mit "00" ersetzt
new_txt = re.sub(pattern, "00", txt)
# new_txt: Das ist ein string mit 00 Zahlen
\end{lstlisting} % code works

\subsection{Library Adapter}\label{LibraryAdapter}
Eine Library durch eine aktuellere austauschen in C++

\subsubsection{MsgPack}\label{msgpack}
library MsgPack
msgpack VS json - Vorteile
joa, hat dann die Ansprüche doch nicht erfüllt *sad trumpet*

\newpage
\section{Ergebnisse}
Präsentiere und diskutiere hier die Ergebnisse deines Berichts.

\newpage
\section{Fazit}
Im Fazit fasst du alles zusammen und gibst einen Ausblick.

\newpage
\bibliographystyle{unsrt}
\bibliography{Praxissemesterbericht_bibliographie}

\end{document}
