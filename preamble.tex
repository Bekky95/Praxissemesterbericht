\usepackage[german]{babel} % deutsch, deutsche Rechtschreibung
\usepackage[utf8]{inputenc} % Unicode-Zeichensatz als Text-Quelle
\usepackage[T1]{fontenc} % Umlaute und deutsches Trennen
\usepackage{mathptmx} % Times New Roman, gewohnter Font
%\usepackage{uarial} % Setzt den gesamten Text in Arial
\usepackage{courier} % einen schickeren Schreibmaschinenfont
\usepackage[scaled=.95]{helvet} % was serifenloses, wenn gebraucht
\usepackage{graphicx} % wir wollen Bilder einfügen
\usepackage{xfrac} % schöne Brüche im Fließtext mit sfrac

% hypersetup selbst ergänzt
\usepackage[dvipsnames]{xcolor}
%basic usage example: \color{ForestGreen}
% Apricot, Aquamarine, Bittersweet, Black, Blue, BlueGreen, BlueViolet, BrickRed,
% Brown, BurntOrange, CadetBlue, CarnationPink, Cerulean, CornflowerBlue, Cyan,
% Dandelion, DarkOrchid, Emerald, ForestGreen, Fuchsia, Goldenrod, Gray, Green,
% GreenYellow, JungleGreen, Lavender, LimeGreen, Magenta, Mahogany, Maroon,
% Melon, MidnightBlue, Mulberry, NavyBlue, OliveGreen, Orange, OrangeRed, Orchid,
% Peach, Periwinkle, PineGreen, Plum, ProcessBlue, Purple, RawSienna, Red, 
% RedOrange, RedViolet, Rhodamine, RoyalBlue, RoyalPurple, RubineRed, Salmon,
% SeaGreen, Sepia, SkyBlue, SpringGreen, Tan, TealBlue, Thistle, Turquoise,
% Violet, VioletRed, White, WildStrawberry, Yellow, YellowGreen, YellowOrange

\usepackage{hyperref}
\hypersetup{
	colorlinks=true,
    linkcolor=RoyalPurple,
    filecolor=ForestGreen,      
    urlcolor=Fuchsia,
    citecolor=BlueViolet
}
% basic colors: red, green, blue, cyan, magenta, yellow, black, gray,
% white, darkgray, lightgray, brown, lime, olive, orange, pink, purple, teal, violet
  
\usepackage{listings} % Schöne Quellcode-Listings [minted wäre besser]

% meine eigenen:
\usepackage{titlesec}
%\usepackage{minted} %ganz zum Schluss auf minted wechseln, 
\usepackage{enumitem} % Hilfreich, um Aufzählungen anzupassen
\usepackage{array}
\usepackage{subcaption}
\usepackage{pdfpages}


\lstset{basicstyle=\sffamily, columns=[l]flexible, mathescape=true, 
  showstringspaces=false, numbers=left, numberstyle=\tiny}
\lstset{language=python} % und nur schöne Programmiersprachen ;-)
% und eine eigene Umgebung für Listings

\usepackage{float} % eigene Fließobjekte, kommen an beliebigen Stellen vor
\newfloat{listing}{htbp}{scl}[section] % Nummeriere je Abschnitt
\floatname{listing}{Listing} % listing ist ein Fließobjekt

% Auch wenn es anrüchig ist, man kann den Platz etwas mehr ausnützen
\usepackage[paper=a4paper,width=15cm,left=30mm,height=22cm]{geometry}
\usepackage{setspace}
\linespread{1.1} % nicht ganz anderthalbzeilig, nur ein bisschen mehr Platz - war 1.15
\setlength{\parskip}{0.4em} % kleiner Paragraphen(Absatz)-abstand
\setlength{\parindent}{0em} % im Deutschen Einrückung nicht üblich

% Seitenmarkierungen 
\usepackage{fancyhdr} % Schickere Header und Footer
\pagestyle{fancy}

% Zeichensatz für Header/Footer
\newcommand{\phv}{\fontfamily{phv}\fontseries{m}\fontsize{9}{11}\selectfont}
\fancyhead[L]{\phv \leftmark} % Kurztitel links oben
\fancyhead[R]{\phv \thepage} % rechts oben die Seitenzahl
\fancyhead[C]{\phv Praxissemesterbericht} % oben Mitte Beschreibung article
\fancyfoot[L]{\phv Hochschule Mannheim} % Institution links unten
\fancyfoot[C]{\ } % keine Seitenzahl unten Mitte
\fancyfoot[R]{\phv Medizintechnik} % Studiengang rechts unten

\usepackage{url} % wir wollen eine URL anzeigen

